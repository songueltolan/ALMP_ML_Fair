%!TEX root = main.tex

\section{Institutional Background}\label{sec:inst}
The German Federal Employment Agency (FEA) finances retraining programmes with the aim of reintegrating otherwise long-term unemployed into the labour market. The focus of this programme is on unemployed who have no vocational training or who have not worked in the occupation they initially trained for in at least four years. The decision on access to the retraining programme is made by the individual's respective local employment agency based on the individual's employment history, educational level, motivation and potential employment prospects. Completion of the retraining programme leads to a state-approved vocational degree. The usual duration of a vocational training programme is three years. However, participants in retraining are expected to complete their training in about a third of the usual time given their previously collected experience in their field of retraining.\\

In 2005 there was a large labour market reform that also affected the allocation of training programmes by the FEA. All changes though the reform aimed at increasing efficiency. Before, the FEA was the only decision-making authority on the allocation of of specific retraining programme providers. With the reform, the unemployed individuals were involved in the decision process by allowing them to decide on the training course conditioned on access to retraining by the FEA. Moreover, after the reform the decision on access to retraining has to be based expressively on potential labour market success of the participant. In addition, training attendance is enforced by the FEA through sanctions \citep{Doerr2017}. The prediction of potential training success and employment prospects for the unemployed is after the reform systematized through annual agency-specific plans that take into account demand for workers in specific occupations as well as prospective working conditions. 