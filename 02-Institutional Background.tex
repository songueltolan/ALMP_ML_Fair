%!TEX root = main.tex

\section{Institutional Background}\label{sec:inst}

\subsubsection{Todo}

\begin{itemize}
\item four types of ALMP\citep{oecd2006,kluve2013aktive}:
\begin{itemize}
    \item Job search support
    \item Training and qualification
    \item Support of employment in the private sector
    \item Employment in the public sector
\end{itemize}
    \item Germany has highest absolute public expenditure in ALMP \citep{kluve2013aktive} (Check OECD)
    \item 

    \item pre-Hartz
    \begin{itemize}
        \item High expenditure levels, long duration of programmes \citep{jacobi2006before}
        \item Mostly training and public job creation measures 
        \item little job search assistance or monitoring by fea, sanctions rarely implemented
        \item law defined eligibility criteria but assignment was caseworker discretion \citep{jacobi2006before}
        \item severe locking-in effects and zero/negative post-treatment effects (Lechner 2000; Caliendo et al. 2003)
    \end{itemize}
    \item 2003-2005 Hartz reform enacted
    \begin{itemize}
        \item reorganisation: results-based accountability and controlling of local employment offices
        \item quasi markets: vouchers for placement services (Vermittlungsgutschein) and training measures (Bildungsgutschein)
        \item standardized profiling (4 types: high to low chance of reemployment) and targeting of programmes to unemployed
        \item Cost-effectiveness in the context of the local labour market is key criterion when assigning programme \citep{jacobi2006before} 
        \item training for best clients and job placement for worst clients \citep{jacobi2006before}
        \item Programme assignment decentralized, allowing for adaptation to local needs
        \item job centre discretion for programme mix and budgeting, but not programme design \citep{harrer2019free}
    \end{itemize}
    \item 1.1.2009 major ALMP reform: Schemes for activation and integration (SAI)
    \begin{itemize}
        \item Gesetz zur Neuausrichtung der arbeitsmarktpolitischen Instrumente
        \item implementation and specific design now determined by local public employment services (PES) while before they were regulated \citep{harrer2019free}
        \item IFT: unpaid internships in firms
        \item SP: placement service focus on improving skills, reducing individual employment impediments and finding work. Replaced previous classrom training. Not limited in duration, except for skills training (8 weeks)
        \item from 2009 to 2016 strong inflow of around 780.000 to SAI programmes \citep{harrer2019free}
    \end{itemize}
    \item 01.04.2012 Gesetz zur Verbesserung der Eingliederungschancen am Arbeitsmarkt
    \item Did 2009 reform make ALMP more efficient by optimizing personalization (heterogenous treatment and heterogeneity in treated: which programme suits which group?)
\end{itemize}

The German Federal Employment Agency (FEA) finances retraining programmes with the aim of reintegrating otherwise long-term unemployed into the labour market. The focus of this programme is on unemployed who have no vocational training or who have not worked in the occupation they initially trained for in at least four years. The decision on access to the retraining programme is made by the individual's respective local employment agency based on the individual's employment history, educational level, motivation and potential employment prospects. Completion of the retraining programme leads to a state-approved vocational degree. The usual duration of a vocational training programme is three years. However, participants in retraining are expected to complete their training in about a third of the usual time given their previously collected experience in their field of retraining.\\

In 2005 there was a large labour market reform that also affected the allocation of training programmes by the FEA. All changes though the reform aimed at increasing efficiency. Before, the FEA was the only decision-making authority on the allocation of of specific retraining programme providers. With the reform, the unemployed individuals were involved in the decision process by allowing them to decide on the training course conditioned on access to retraining by the FEA. Moreover, after the reform the decision on access to retraining has to be based expressively on potential labour market success of the participant. In addition, training attendance is enforced by the FEA through sanctions \citep{Doerr2017}. The prediction of potential training success and employment prospects for the unemployed is after the reform systematized through annual agency-specific plans that take into account demand for workers in specific occupations as well as prospective working conditions. 