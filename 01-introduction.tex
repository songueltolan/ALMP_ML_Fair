%!TEX root = main.tex
\section{Introduction}

The investment in training programmes for the unemployed, take up a large share in labour market spendings in many countries in the Organisation for Economic Co-operation and Development (OECD). For instance, in 2014 EU countries spent on average about 10\% of their labour market budget on training programmes for the unemployed.\footnote{see \href{https://data.oecd.org/socialexp/public-spending-on-labour-markets.htm}{OECD 2014}} The German labour market is highly structured by occupations that require specific vocational training degrees which is why vocational training programmes represent the largest share of its active labour market policies. Knowing the effect of these programmes is crucial for the efficiency of labour market policies for the unemployed.  \\
There is an extensive literature that analyses employment effects of active labour market policies \citep[see][for overviews]{Kluve2010, Card2017} and some studies that look at the employment effects of training programmes in Germany \citep{Huber2018,Kruppe2018}. In a more recent study, \citet{Knaus2017} apply machine learning to the analysis of this research question. We contribute to this recently developing literature that combines econometric methods with machine learning techniques for more robust causal inference \citep{Belloni2014,Athey2015,Athey2017}.\\




